\section{Теорема о структуре спектра компактного оператора}
\label{sec:q-45}

Вспомним определения собственного значения оператора и спектра.

Пусть $A$ --- линейный оператор. $\lambda$ --- \textit{собственное значение} оператора $A$, если уравнение $Ax = \lambda x$ имеет ненулевое решение, при этом $x$ называется \textit{собственным элементом}.

\textit{Спектр оператора} --- это совокупность всех собственных значений: $\operatorname{Sp}A = \{\lambda : \exists x \in X : Ax = \lambda x\}$.

\subsubsection*{Теорема}
Спектр компактного оператора конечен или счётен. Его непрерывный спектр либо пуст, либо состоит из нуля. Ненулевые собственные значения имеют конечную кратность, причем, если их бесконечно много, то они образуют последовательность, сходящуюся к нулю.\\
Доказательство похожей теоремы есть в учебнике \cite[с.~244]{kolmogorov}.\\

Ниже приводится серия теорем, связанных со спектром компактного оператора. Источник: \cite{pnu-compact-operators}.

\subsubsection*{Теорема}
Пусть $A$ --- компактный линейный оператор, $X_\lambda$ --- пространство собственных векторов (собственное подпространство), соответствующее $\lambda \neq 0$. Тогда $X_\lambda$ конечномерно.

\subsubsection*{Теорема}
Всякий компактный оператор в банаховом пространстве имеет конечное число линейно независимых элементов, соответствующих собственному
значению $\lambda \neq 0$.

\subsubsection*{Теорема}
Пусть $A$ --- компактный линейный оператор. Если $A \neq 0$, то $A$ имеет хотя бы одно отличное от нуля собственное значение.

\subsubsection*{Теорема}
Все собственные значения $\lambda$ компактного непрерывного самосопряженного оператора $A$ вещественны и
$$\inf_{\| x \| = 1}(Ax, x) \leqslant \lambda \leqslant \sup_{\Vert x \Vert = 1}(Ax, x).$$

\subsubsection*{Следствие}
Если $A$ --- компактный самосопряженный оператор, то $\|A\| = max_j\{|\lambda_j|\}$.

\subsubsection*{Теорема (Гильберта--Шмидта)}
Для любого компактного самосопряженного оператора в гильбертовом пространстве существует ортогональная система собственных элементов $\{\varphi_n\}_n$, соответствующих собственным значениям
$\lambda_n \neq 0$: для любого элемента $x \in H$ имеет место представление
$$x = \sum_{n}c_n\varphi_n + x',$$
где $x' \in \operatorname{Ker}A$, при этом
$$Ax = \sum_{n}\lambda_nc_n \varphi_n$$
Доказательство теоремы: см. учебник \cite[с.~246]{kolmogorov}.

Напомним, что ядром оператора $A$ в пространстве $X$ называется множество всех векторов $x \in X$ таких, что $Ax = 0$. Ядро оператора обозначается как $\operatorname{Ker}A$.

Замечание. Если система $\{\varphi_n\}_n$ бесконечна, то
$$\lim_{n \to \infty}\lambda_n = 0.$$