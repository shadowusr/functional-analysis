\section{Первая и вторая теоремы Фредгольма}
\label{sec:q-43}

Пусть $A$ --- компактный линейный оператор, действующий в банаховом пространстве $X$. Рассмотрим следующие уравнения:
\begin{equation}
	x - Ax = y,
	\label{eq:43-1}
\end{equation}
соответствующее ему однородное уравнение:
\begin{equation}
	z - Az = 0,
	\label{eq:43-2}
\end{equation}
сопряженное уравнение:
\begin{equation}
	f - A^*f = \omega
	\label{eq:43-3}
\end{equation}
и сопряженное однородное уравнение:
\begin{equation}
	\psi - A^*\psi = 0.
	\label{eq:43-4}
\end{equation}

\subsubsection*{Теорема (первая теорема Фредгольма)}
Пусть $A$ --- линейный компактный оператор в банаховом пространстве $X$. Следующие условия эквивалентны:
\begin{enumerate}
	\itemsep0pt
	\item Уравнение (\ref{eq:43-1}) имеет решение при любой правой части $y$;
	\item уравнение (\ref{eq:43-2}) имеет только тривиальное решение;
	\item уравнение (\ref{eq:43-3}) имеет решение при любой правой части $\omega$;
	\item уравнение (\ref{eq:43-4}) имеет только тривиальное решение.
\end{enumerate}
Если выполнено одно из условий 1--4, то операторы $I - A$ и $I - A^*$ непрерывно обратимы (т. е. имеют обратный оператор, заданный на всей области определения).\\
Доказательство: см. учебник \cite[с.~212]{trenogin}.

\subsubsection*{Теорема (вторая теорема Фредгольма)}
Пусть $A$ --- линейный компактный оператор в $X$. Тогда уравнения (\ref{eq:43-2}) и (\ref{eq:43-4}) имеют одинаковое конечное число линейно независимых решений.\\
Доказательство: см. учебник \cite[с.~213]{trenogin}.
