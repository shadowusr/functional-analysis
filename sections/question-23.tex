\section{Разложение гильбертова пространства в ортогональную сумму двух подпространств}
\label{sec:q-23}

Два элемента называются \textit{ортогональными}, если их скалярное произведение равно нулю, а сами элементы являются ненулевыми.

Два подмножества $M$ и $N$ гильбертова пространства \textit{ортогональны} $(M \perp N)$, если любые два элемента $f \in M,\;g \in N$ ортогональны.

\textit{Подпространство} --- это подмножество некоторого пространства, которое само является пространством соответствующего типа со свойствами, индуцированными объемлющим пространством.

Множество всех элементов пространства, ортогональных некоторому подмножеству $A$, является подпространством и называется \textit{ортогональным дополнением} этого множества.

Пусть $L$ --- некоторое подпространство в гильбертовом пространстве $H$. Тогда для любого элемента $f \in H$ справедливо единственное разложение $f=g+h$, где $g \in L$, а $h \perp L$. Элемент $g$ называется \textit{проекцией} элемента $f$ на $L$. Совокупность элементов $h$, ортогональных подпространству $L$ образует (замкнутое) подпространство $M$, являющееся ортогональным дополнением подпространства $L$.

Говорят, что пространство $H$ разложено \textit{в прямую сумму} подпространств $L$ и $M$, что записывается как $H=L \oplus M$. Аналогично можно записать $L=H \ominus M$.

Дополнительная информация приводится в учебнике \cite[с.~53]{trenogin}.