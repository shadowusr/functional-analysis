\section{Норма отображения}
\label{sec:q-10}
Вспомним определение ограниченного отображения:\\
Линейное отображение $A:X\to Y$ называется \textit{ограниченным}, если существует такое положительное число $C$, что $\|Ax\| \le C\|x\|$ для любого $x$ из $D(A)$.

Тогда \textit{норма отображения} --- это наименьшее из таких чисел $C$. Норма отображения обозначается как $\Vert A \Vert$. Формально норма отображения определяется как
$$\Vert A \Vert = \sup_{x\neq 0}\frac{\Vert Ax \Vert}{\Vert x \Vert}$$

% todo: добавить информацию из конспекта
% todo: добавить эквивалентные определения нормы