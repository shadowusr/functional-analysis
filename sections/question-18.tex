\section{Спектральный радиус оператора}
Материал этого раздела опирается на информацию из \hyperref[sec:q-17]{вопроса 17}.

\textit{Спектральный радиус оператора} --- это максимальный по модулю элемент его спектра. Спектральный радиус оператора $A$ обозначается как $r_\sigma(A)$. При этом выполняется равенство:
$$r_\sigma(A) = \lim_{n \rightarrow \infty} \Vert A^n \Vert ^{1/n}$$
Здесь под $A^n$ подразумевается возведение матрицы оператора $A$ в степень.

\subsubsection*{Теорема}
Пусть $A\in L(X)$, $X$ --- комплексное банахово пространство. Тогда существует конечный предел $\displaystyle r_\sigma(A) = \lim_{n \rightarrow \infty} \Vert A^n \Vert ^{1/n}$ и выполняется соотношение $r_\sigma(A) \leqslant \Vert A \Vert$ \cite[с.~249]{trenogin}.\\
Подробнее о пространстве $L(X)$: \hyperref[sec:q-11]{см. вопрос 11}.

Дополнительные теоремы, связанные со спектральным радиусом, доступны в учебнике \cite[с.~249]{trenogin}.