\section{Подпространство $\delta_\infty (x, y)$ --- множество компактных операторов}
\label{sec:q-39}
Обозначим множество всех компактных операторов из $L(X, Y)$ как $\sigma(X, Y)$.

\subsubsection*{Теорема}
$\sigma(X, Y)$ является замкнутым подпространством в $L(X, Y)$.

\subsubsection*{Теорема}
Если $X$ или $Y$ конечномерно, то $\sigma(X, Y) = L(X, Y)$.

\subsubsection*{Следствие}
Всякий линейный функционал $f \in X^*$ компактен (достаточно вспомнить, что $f$ переводит $X$ в одномерное пространство).

\subsubsection*{Следствие}
Если $\displaystyle A = \lim_{n \rightarrow \infty} A_n$ (в метрике $L(X, Y)$), где  $A_n$ компактны или конечномерны, то $A$ компактен.

\subsubsection*{Теорема}
Пусть $A \in L(X, Y), B \in L(Y, Z)$. Если хоть один из этих операторов компактен, то компактным будет и их произведение $BA$.

Доказательства теорем и подробная информация доступна в учебнике \cite[с.~204]{trenogin}.

\subsubsection*{Утверждение}
Для любого банахова пространства $X$ пространство $\sigma(X,Y)$ является замыканием пространства конечномерных операторов из $X$ в $Y$.