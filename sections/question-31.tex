\section{Оператор Фредгольма и его свойства}
\label{sec:q-31}
\textit{Интегральный оператор Фредгольма} --- это компактный линейный интегральный оператор вида
$$(Af)(x) = \int_{G} K(x, t) f(t) dt,$$
отображающий одно пространство функций в другое. Здесь $G$ --- область в евклидовом пространстве $\mathbb{R}^n$, $K(x, t)$ --- функция, заданная на декартовом квадрате $G \times G$, называемая \textit{ядром интегрального оператора}. Интегральный оператор Фредгольма и его свойства используются при решении интегрального уравнения Фредгольма.

Свойства:
\begin{enumerate}
	\itemsep0pt
	\item Линейность.\\
	Интегральный оператор Фредгольма является линейным, то есть выполняется $A(\alpha f + \beta g) = \alpha A f + \beta A g$.
	\item Непрерывность.
	Так, оператор с $L_2$--ядром:
	$$\int_G\int_G |K(x, y)|^2 dx dy \leqslant N^2 < \infty$$
	Переводит $L_2(G)$ в $L_2(G)$, непрерывен и удовлетворяет оценке:
	$$\Vert Af\Vert_{L_2} \leqslant N\Vert f \Vert_{L_2}.$$
	Существуют условия непрерывности интегральных операторов из $L_{p}$ в ${\displaystyle L_{q}}$.
	\item Компактность (вполне непрерывность).
	Интегральный оператор с непрерывным ядром является пределом последовательности конечномерных операторов с вырожденными ядрами.
\end{enumerate}
Источник: \cite{fredholm-integral-operator}.

Под оператором Фредгольма может также пониматься класс фредгольмовых операторов, т. е. линейных операторов, у которых ядро и коядро конечномерны.\\
Подробная информация об этой теме доступна на википедии \cite{fredholm-operator} и в учебнике \cite[с.~219]{trenogin}.
% todo: написать конспект по фредгольмовым операторам