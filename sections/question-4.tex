\section{Принцип сжимающих отображений}
Отображение $\Phi$ метрического пространства $M$ в себя называется сжимающим (с коэффициентом $q < 1$), если $\rho(\Phi x,\Phi y) \leqslant q\rho(x,y)\; \forall x,y \in M$, здесь $\rho$ --- функция расстояния.

Точка $x$ называется \textit{неподвижной} относительно отображения $f$, если выполняется $f(x) = x$.

\subsubsection*{Теорема (принцип сжимающих отображений)}
Сжимающее отображение метрического пространства в себя имеет неподвижную точку, причем ровно одну.\\

Формулировка теоремы из учебника: Пусть $X$ --- банахово пространство, $Q$ --- замкнутое в $X$ множество, $\Phi$ --- сжимающий оператор, действующий из $Q$ в $Q$.\\
Тогда оператор $\Phi$ имеет в $Q$ единственную неподвижную точку $x^*$.\\
Пусть $x_0 \in Q$. Образуем последовательность
$$x_n = \Phi(x_{n-1}), \qquad n = 1, 2, \dots$$
Тогда $\{x_n\} \subset Q$ и $x_n \to x^*$ при $n \to \infty$. Кроме того, справедлива оценка скорости сходимости
$$\| x_n - x^*\| \leqslant \frac{q^n}{1 - q}\|\Phi (x_0) - x_0\|.$$
Доказательство: см. \cite[с.~381]{trenogin}.

\subsubsection*{Теорема}
Пусть оператор $\Phi$ отображает замкнутое множество $Q \subset X$ в себя и при некотором натуральном $m$ оператор $\Phi^m$ является на $Q$ сжатием. Тогда в $Q$ существует единственная неподвижная точка $x^*$, начиная с любого начального приближения $x_0 \in Q$.\\
Доказательство: см. \cite[с.~382]{trenogin}.

