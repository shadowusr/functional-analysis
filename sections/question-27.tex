\section{Сопряженный оператор и его свойства}
\label{sec:q-27}

Пусть $U$ --- унитарное пространство, $A$ --- линейный оператор в $U$. Оператор $A^*$ называется \textit{сопряженным} по отношению к линейному оператору $A$, если для любых векторов $x, y \in U$ выполняется равенство
$$(Ax, y) = (x, A^*y).$$

\subsubsection*{Теорема}
Сопряженный оператор $A^*$ обладает следующими свойствами:
\begin{enumerate}
	\itemsep0pt
	\item $A^*$ --- линейный оператор;
	\item $(A + B)^* = A^* + B^*$;
	\item $(\alpha A)^* = \bar{\alpha}A^*$;
	\item $(AB)^* = B^*A^*$; 
	\item $(A^*)^* = A$.
\end{enumerate}
Доказательство: \cite[с.~1]{adjoint-operator}.

\subsubsection*{Лемма}
Если $A \in L(X, Y)$, то $\Vert A^* \Vert = \Vert A \Vert$\cite[с.~179]{trenogin}.