\section{Компактность и конечномерность}
\label{sec:q-37}
% todo: Уточнить информацию и найти подробности
Любой линейный ограниченный оператор с конечномерным образом является компактным (такие операторы называются \textit{конечномерными}).

Для компактного оператора ${\displaystyle T:X\to Y}$, где $Y$ --- гильбертово пространство, всегда существует последовательность конечномерных операторов, сходящаяся к $T$ по норме. Однако, это неверно для произвольного пространства $Y$.

Говорят, что банахово пространство $Y$ обладает свойством \textit{аппроксимации}, если для любого банахова пространства $X$ любой компактный оператор ${\displaystyle T:X\to Y}$ может быть приближен конечномерными операторами. Существуют сепарабельные банаховы пространства не обладающие свойством аппроксимации.

Обратимый оператор ${\displaystyle A:X\to Y}$ компактен тогда и только тогда, когда $X,Y$ конечномерны.

Множество называется \textit{вполне ограниченным}, если для любого положительного $\varepsilon$ существует конечная $\varepsilon$--сеть для этого множества.

\subsubsection*{Утверждение}
Пусть $X$ --- подмножество в $\mathbb{R}^n$ (где $\mathbb{R}^n$ снабжено евклидовой нормой). Тогда:
\begin{itemize}
	\itemsep0pt
	\item $X$ вполне ограничено тогда и только тогда, когда оно ограничено.
	\item$ X$ компактно тогда и только тогда, когда оно замкнуто и ограничено.
\end{itemize}
Эти факты справедливы для подмножеств любого конечномерного нормированного пространства.\\
Доказательство: см. \cite{fa-lectures-18}.