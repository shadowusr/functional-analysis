\section{Ортопроектор и его свойства}
\label{sec:q-28}

\textit{Проектор} --- это такой линейный оператор $P$, действующий в линейном пространстве, что $P^2=P$. То есть применив этот оператор дважды, мы получим тот же результат, какой мы бы получили при применении его один раз. Проектор при применении к образу оставляет его без изменений.

Элемент $y_0 \in L$ называется \textit{ортогональной проекцией} элемента $x_0 \in E$ в подпространство $L$, если
$$(x_0-y_0,y)=0\;\forall y \in L.$$
Любой элемент $x \in E$ может иметь только одну проекцию на данное подпространство $L$.

Пусть в гильбертовом пространстве $H$ задано подпространство $M$. Каждому $x \in H$ можно поставить в соответствие элемент $y \in M$ --- ортогональную проекцию  $x$ на $M$ (по т. Рисса). Тем самым определен \textit{оператор ортогональной проекции} $P$: $y = Px$.

Свойства ортогональных проекторов:
\begin{enumerate}
	\itemsep0pt
	\item Каждый ортопроектор $P$ является всюду определенным в $H$ линейным оператором со значениями в $H$.
	\item $P \in L(H)$, причем $\Vert P \Vert = 1$, если $M \neq \{0\}$.
	\item $P^2 = P$.
	\item $P$ самосопряжен (\hyperref[sec:q-29]{см. вопрос 29}).
\end{enumerate}
Подробнее: см. учебник \cite[с.~187]{trenogin}.

\subsubsection*{Теорема}
Если подпространство $L$ евклидова пространства $E$ полное, то у любого элемента $x \in E$ существует ортогональная проекция на подпространство $L$.
Доказательство: см. \cite[с.~95]{mipt-lectures-1}.