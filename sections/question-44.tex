\section{Третья теорема Фредгольма}
\label{sec:q-44}

Пусть $A$ --- компактный линейный оператор, действующий в банаховом пространстве $X$. Рассмотрим следующие уравнения:
\begin{equation}
x - Ax = y,
\label{eq:44-1}
\end{equation}
соответствующее ему однородное уравнение:
\begin{equation}
z - Az = 0,
\label{eq:44-2}
\end{equation}
сопряженное уравнение:
\begin{equation}
f - A^*f = \omega
\label{eq:44-3}
\end{equation}
и сопряженное однородное уравнение:
\begin{equation}
\psi - A^*\psi = 0.
\label{eq:44-4}
\end{equation}

\subsubsection*{Теорема (третья теорема Фредгольма)}
Пусть $A$ --- линейный вполне непрерывный оператор в $X$. Для того, чтобы уравнение (\ref{eq:44-1}) имело хоть одно решение, необходимо и достаточно, чтобы для любого решения $\psi$ уравнения (\ref{eq:44-4}) выполнялось условие $\langle y, \psi \rangle = 0$.\\
Доказательство: см. учебник \cite[с.~215]{trenogin}.

Данную теорему можно дополнить таким утверждением: чтобы уравнение (\ref{eq:44-2}) имело хоть одно решение, необходимо и достаточно, чтобы для всех решений $\varphi$ уравнения (\ref{eq:44-3}) выполнялось условие $\langle \varphi, \omega \rangle$

Резюмируем полученные результаты:
\begin{enumerate}
	\itemsep0pt
	\item Оператор $I - A$ непрерывно обратим, тогда (\ref{eq:44-1}) имеет при любой правой части $y$ единственное решение $x = (I - A)^{-1}y$.
	\item $N(I - A) \neq {0}$; если $\langle y, \psi \rangle \neq 0$ хоть для одного решения $\psi$ сопряженного однородного уравнения (\ref{eq:44-4}), то (\ref{eq:44-1}) решений не имеет.
	\item $N(I - A) \neq {0}$; если $\langle y, \psi \rangle = 0$ для всех решений (\ref{eq:44-4}), то общее решение уравнения (\ref{eq:44-1}) имеет вид
	$$x = x_0 + \sum_{i = 1}^{n}\xi_i\varphi_i,$$
	где $x_0$ --- частное решение (\ref{eq:44-1}), $\{\varphi_i\}_1^n$ --- базис подпространства решений уравнения (\ref{eq:44-3}), а $n$ --- размерность этого подпространства.
\end{enumerate}