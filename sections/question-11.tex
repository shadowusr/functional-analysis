\section{Пространство $L(x, y)$ и условие его полноты}
\label{sec:q-11}
Пусть заданы нормированные пространства $X$, $Y$, а $A, B, C, \dots$ --- линейные непрерывные отображения, действующие из $X$ в  $Y$. Определим на множестве всевозможных таких отображений операции сложения отображений и умножения отображения на число:
$$(A+B)(x)=A(x)+B(x),$$
$$(\lambda\cdot A)(x)=\lambda\cdot A(x).$$

Теперь на полученном линейном пространстве зададим норму как
 $$\Vert A \Vert = \sup_{\Vert x \Vert \leqslant 1} \Vert Ax \Vert.$$
 
 Полученное нормированное пространство линейных непрерывных отображений, действующих из всего $X$ в $Y$, обозначим как $L(x, y)$ \cite[с.~118]{trenogin}.

 В случае, когда $X = Y$, будем писать $L(X)$.
 
 \subsubsection*{Теорема (о полноте пространства $L(X,Y)$)}
 Если $Y$ --- полное, то и $L(X, Y)$ --- полное.
 
 Эквивалентная формулировка: если $Y$ --- банахово, то и $L(X, Y)$ --- банахово \cite[с.~120]{trenogin}. Подробнее о банаховых пространствах: \hyperref[sec:q-2]{см. вопрос 2}.