\section{Теорема Банаха об обратном операторе. Спектр оператора и его классификация}
\label{sec:q-17}

\subsubsection*{Теорема (Банаха об обратном операторе)}
Если $A$ --- ограниченный линейный оператор, отображающий взаимно однозначно $X$ на $Y$, $X$ и $Y$ --- банаховы пространства, то обратный оператор $A^{-1}$ существует и ограничен \cite[с.~128]{trenogin}. В учебнике приводится доказательство более сильного варианта этой теоремы \cite[с.~159]{trenogin}.\\
Подробнее об ограниченных операторах: \hyperref[sec:q-9]{см. вопрос 9}.\\

\textit{Спектром оператора} $A$ называется дополнение к $\rho(A)$ в комплексной плоскости. Спектр обозначается как $\sigma(A)$ \cite[с.~248]{trenogin}.

Проще говоря, \textit{спектр оператора} --- это множество точек, которые не входят в его резольвентное множество. Подробнее о резольвентных множествах: \hyperref[sec:q-11]{см. вопрос 16}.\\

Приведем несколько определений, необходимых для проведения классификации спектров.

Подмножество $B$ множества $A$ называется \textit{правильным подмножеством}, если $B \neq A$, то есть существует хотя бы один элемент из $A$, который не принадлежит $B$.

Подмножество $A$ топологического пространства $X$ называется \textit{плотным}, если в произвольной окрестности каждой точки $x \in X$ содержится элемент из $A$.\\

Внутри спектра оператора можно выделять части, не одинаковые по своим свойствам. Одной из основных классификаций спектра является следующая \cite{spectrum-decomposition}:
\begin{enumerate}
	\itemsep0em
	\item \textit{дискретным (точечным) спектром} $\sigma_p(A)$ называется множество всех собственных значений оператора $A$; в конечномерном случае присутствует только точечный спектр;
	\item \textit{непрерывным спектром} $\sigma_c(A)$ называется множество значений $\lambda$, которые не являются собственными значениями оператора $A$, но множество значений оператора $A - \lambda I$ является правильным плотным подмножеством пространства, на котором задан оператор $A$;
	\item \textit{остаточным спектром} $\sigma_r(A)$ называется множество точек спектра, не входящих ни в дискретную, ни в непрерывную части.
\end{enumerate}