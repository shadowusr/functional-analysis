\section{Теорема Банаха---Штейнгауза}

Будем рассматривать последовательность линейных ограниченных отображений $A_n:X\rightarrow Y$.

Последовательность $A_n$ \textit{поточечно ограничена}, если $\forall x \in X$ выполняется $$ \sup_{n \in \mathbb{N}}\Vert A_nx \Vert < +\infty$$

Последовательность $A_n$ \textit{равномерно ограничена}, если
$$\sup_{n \in \mathbb{N}} \Vert A_n \Vert < +\infty$$

\subsubsection*{Теорема (Банаха---Штейнгауса)}
Пусть $X$ --- банахово, $L(X, Y)$ --- нормированное пространство линейных непрерывных отображений, $A_n \in L(X, Y)$. Тогда если $A_n$ поточечно ограничена, то $A_n$ равномерно ограничена \cite{banach-steingaus}.\\
Теорема также приводится в учебнике \cite[с.~123]{trenogin}. Подробнее о нормированных пространствах: \hyperref[sec:q-5]{см. вопрос 5}, о пространстве $L(X, Y)$: \hyperref[sec:q-11]{см. вопрос 11}.
%https://neerc.ifmo.ru/wiki/index.php?title=Теорема_Банаха-Штейнгауза