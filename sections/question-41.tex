\section{Компактность оператора Фредгольма}
\label{sec:q-41}
Рассмотрим оператор $A\varphi = \psi$ вида
\begin{equation}
	\int\displaylimits_a^b K(s, t) \varphi(t) dt = \psi(s), \label{eq:41-1}
\end{equation}
называемый \textit{оператором Фредгольма}.

\subsubsection*{Теорема}
Равенство (\ref{eq:41-1}), где $K(s, t)$ --- квадратично интегрируемая функция, определяет в пространстве $L_2[a, b]$ компактный линейный оператор $A$, норма которого удовлетворяет неравенству
$$\|A\| \leqslant \sqrt{\int\displaylimits_a^b\int\displaylimits_a^b\|K(s, t)|^2 ds \; dt}.$$
Доказательство: см. \cite[с.~461]{kolmogorov}.

Другой вариант теоремы с более коротким и понятным доказательством:
\subsubsection*{Теорема}
Пусть $A$ --- оператор Фредгольма, действующий из $h[a,b]$ в $h[a,b]$. Тогда $А$ --- вполне непрерывный (компактный) оператор.\\
Доказательство: \cite[с.~13]{msu-fa-2}.\\
В учебнике В. А. Треногина также есть доказательство компактности оператора Фредгольма: см. \cite[с.~208]{trenogin}.