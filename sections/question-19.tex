\section{Сопряженное пространство. Пример}
\textit{Линейный функционал} --- это линейное отображение, действующее из линейного пространства $L$ над полем $K$ в поле $K$.\\
Подробнее о линейных отображениях: \hyperref[sec:q-8]{см. вопрос 8}, о линейном пространстве и поле: \hyperref[sec:q-5]{см. вопрос 5}.

Пусть $X$ --- банахово пространство, $X^1$ --- вещественная ось, если $X$ вещественно, и комплексная плоскость, если $X$ комплексно.

Рассмотрим $L(X, X^1)$ --- пространство линейных ограниченных функционалов, заданных на $X$. Это пространство называется \textit{сопряженным} к $X$ и обозначается $X^*$ \cite[с.~169]{trenogin}.\\
Подробнее о пространстве $L(X)$: \hyperref[sec:q-11]{см. вопрос 11}.\\

Будем обозначать значение линейного функционала $f \in X^*$ на элементе $x \in X$ как $\langle x,f\rangle$ (иногда используется обозначение $f(x)$).

Напомним, что норма линейного функционала $f$ определяется по формуле
$$\Vert f \Vert = \sup_{\Vert x \Vert \leqslant 1}| \langle x, f\rangle|.$$

В сопряженном пространстве $X^*$, как и в любом пространстве линейных ограниченных операторов $L(X, Y)$, можно ввести два вида сходимости последовательностей.

\textit{Сходимость по норме} или \textit{сильная сходимость} определяется следующим образом: $f_n \rightarrow f,\; n \rightarrow \infty\;(f_n, f \in X^*)$, если $\Vert f_n - f \Vert \rightarrow 0, \; n \rightarrow \infty$. То есть $f_n$ сходится по норме к $f$, если норма их разности стремится к нулю при $n \rightarrow \infty$.

Определим теперь \textit{*-слабую сходимость}: $f_n \rightarrow f$ *-слабо при $n \rightarrow \infty$, если для всех $x \in X$ выполняется $\langle x, f_n \rangle \rightarrow \langle x, f \rangle$ при $n \rightarrow \infty$ \cite[с.~170]{trenogin}.

\subsubsection*{Пример}
Пусть $c_0$ --- линейное пространство бесконечно малых последовательностей $x = \{\xi_n\}\;(\xi_n \rightarrow 0,\; n \rightarrow \infty)$. Норму в $c_0$ зададим по формуле $\displaystyle \Vert x \Vert = \sup_{n}|\xi_n|$.

Тогда $(c_0)^* = l_1$ --- пространство последовательностей $a = \{\alpha_n\}$, таких, что $\displaystyle \Vert a \Vert_1 = \sum_{n = 1}^{\infty}|\alpha_n|<+\infty$ \cite[с.~170]{trenogin}.