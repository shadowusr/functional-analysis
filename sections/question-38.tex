\section{Компактные операторы и их свойства}
\label{sec:q-38}
Пусть $X, Y$ --- банаховы пространства. Линейный оператор $T: X \rightarrow Y$ называется \textit{компактным} (или \textit{вполне непрерывным}), если любое ограниченное подмножество в $X$ он переводит в предкомпактное подмножество в $Y$.

Основные свойства компактных операторов:
\begin{enumerate}
	\itemsep0pt
	\item Всякий компактный оператор является ограниченным, однако не всякий ограниченный оператор является компактным.
	\item Линейная комбинация компактных операторов $A, B$ вида $\alpha A + \beta B$, где $\alpha, \beta$ --- числа, также является компактным оператором.
	\item Пусть $A$ --- вполне непрерывный оператор, отображающий бесконечномерное банахово пространство в себя, и $B$ --- произвольный линейный ограниченный оператор, определённый на этом же пространстве. Тогда $AB$ и $BA$ являются компактными операторами.
	\item Если последовательность компактных операторов $\left \{ A_{n} \right \}$, отображающих пространство $E_{x}$ в полное пространство $E_{y}$, равномерно сходится к оператору $A$ (то есть $\left \| A - A_{n} \right \|  \to 0$), то $A$ также компактный оператор.
	\item Если оператор компактен, то сопряженный к нему тоже компактен.
\end{enumerate}

\subsubsection*{Теорема}
Если $A \in L(X, Y)$ компактен, то любое ограниченное множество из $X$ он переводит во множество, компактное в $Y$.

Доказательства теорем и подробная информация доступна в учебнике \cite[с.~203]{trenogin}.

\subsubsection*{Теорема}
Спектр компактного оператора конечен или счётен. Его непрерывный спектр либо пуст, либо состоит из нуля. Ненулевые собственные значения имеют конечную кратность, причем, если их бесконечно много, то они образуют последовательность, сходящуюся к нулю.

\subsubsection*{Примеры}
\begin{itemize}
	\itemsep0pt
	\item Любой ограниченный линейный оператор, действующий между банаховыми пространствами и имеющий конечномерную область значений, компактен.
	\item Тождественный оператор $I$ компактен тогда и только тогда, когда пространство конечномерно.
	\item Пусть $\Omega$ --- связное открытое подмножество $\mathbb{R}^n$ и ядро $k : \Omega \times \Omega \rightarrow \mathbb{R}$, тогда оператор $T$ на пространстве $L^2(\Omega, \mathbb{R})$, определенный как
	$$ (Tf)(x) = \int_{\Omega} k(x, y)f(y)dy,$$
	это компактный оператор. Здесь $L^2$ --- множество квадратно интегрируемых функций, то есть таких функций, что $\displaystyle \int_{-\infty}^{+\infty}|f(x)|^2 dx < \infty$.
\end{itemize}