\section{Основные леммы теории компактных операторов}
\label{sec:q-42}
В этом разделе приводятся леммы и теоремы из соответствующего раздела учебника В. А. Треногина\cite[с.~203]{trenogin}. Однако, формулировка вопроса весьма размыта, поэтому составить исчерпывающий конспект по этому разделу без уточнений невозможно.

\subsubsection*{Теорема}
Если $A \in L(X, Y)$ компактен, то любое ограниченное в $X$ множество он переводит во множество, компактное в $Y$.

\subsubsection*{Теорема}
Множество всех компактных операторов $\sigma(X, Y)$ является подпространством в $L(X, Y)$.

\subsubsection*{Теорема}
Если $X$ или $Y$ конечномерно, то $\sigma(X, Y) = L(X, Y)$.

\subsubsection*{Следствие}
Всякий линейный функционал $f \in X^*$ компактен (т. к. $f$ переводит $X$ в одномерное пространство).

\subsubsection*{Следствие}
Если $\displaystyle A = \lim_{n \to \infty} A_n$ (в метрике $L(X, Y)$), где $A_n$ компактны или конечномерны, то $A$ вполне непрерывен.

\subsubsection*{Теорема}
Пусть $A \in L(X, Y),\; B \in L(Y, Z)$. Если хоть один из этих операторов компактен, то компактным будет и их произведение $BA$.

\subsubsection*{Лемма}
Если последовательность слабо сходится и компактна, то она сходится.

\subsubsection*{Теорема}
Пусть $A \in \sigma(X, Y)$. Если $x_n \to x_0, \; n\to \infty$, слабо, то $Ax_n \to Ax_0, \; n \to \infty$.

\subsubsection*{Теорема (Шаудера)}
Пусть $A \in L(X, Y)$, где $Y$ --- плотное. Оператор $A$ компактен тогда и только тогда, когда $A^*$ компактен.

\subsubsection*{Теорема}
Пусть $A$ --- линейный компактный оператор, $X$ --- банахово пространство. Тогда множества значений операторов $I - A$ и $I - A^*$ замкнуты и, значит, являются подпространствами в $X$ и в $X^*$ соответственно.