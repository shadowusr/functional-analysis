\section{Виды операторной сходимости}
Пусть дана последовательность операторов $\{A_n\}\subset L(X, Y)$. Будем говорить, что она сходится \textit{равномерно} $A_n \rightarrow A \in L(X, Y)$ при $n \rightarrow \infty$, если $\Vert A_n - A\Vert \rightarrow 0$ при $n \rightarrow \infty$.

\subsubsection*{Теорема}
Чтобы $A_n \rightarrow A$ при $n \rightarrow \infty$ равномерно ($A_n, A$ принадлежат $L(X, Y)$), необходимо и достаточно, чтобы $A_nx \rightarrow Ax$ равномерно при $n \rightarrow \infty, \Vert x \Vert \leqslant 1$ \cite[с.~119]{trenogin}.

Пусть дана последовательность $\{A_n\}\subset L(X, Y)$. Будем говорить, что эта последовательность \textit{сильно сходится} к оператору $A \in L(X, Y)$, если для любого $x \in X$
$$\Vert A_nx - Ax\Vert \rightarrow 0,\;n \rightarrow \infty.$$

Подробная информация о видах сходимости есть в учебнике \cite[с.~122]{trenogin}. Подробнее о пространстве $L(X, Y)$: \hyperref[sec:q-11]{см. вопрос 11}.