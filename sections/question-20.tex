\section{Теорема Хана---Банаха и её следствие}
Множество $\widetilde{E}$ в линейном пространстве $E$ называется \textit{линейным многообразием}, если для любых $x, y \in \widetilde{E}$ и любых скаляров $\lambda, \mu$ линейная комбинация $\lambda x + \mu y$ принадлежит $\widetilde{E}$ \cite[с.~13]{trenogin}.

\subsubsection*{Теорема (Хана---Банаха)}
Пусть в вещественном нормированном пространстве $X$ задан линейный ограниченный функционал $f$ с $D(f) \subset X$. Тогда существует линейный ограниченный функционал $\tilde{f}$, определенный на всём $X$, причем $\Vert \tilde{f} \Vert = \Vert f \Vert$ и $\langle x,\tilde{f}\rangle = \langle x,f \rangle$ для всех $x \in D(f)$\cite[с.~163]{trenogin}.\\
Подробнее об ограниченности операторов: \hyperref[sec:q-9]{см. вопрос 9}. Запись $\langle x,f \rangle$ означает значение функционала $f$ в точке $x$.

\subsubsection*{Следствие}
Пусть $X$ --- нормированное пространство и $x \in X,\;x\neq 0$. Тогда существует всюду заданный в $X$ линейный ограниченный функционал $f$ такой, что $\Vert f \Vert = 1,\; \langle x, f \rangle = \Vert x \Vert$\cite[с.~167]{trenogin}.

\subsubsection*{Следствие}
Пусть в нормированном пространстве $X$ задано линейное многообразие $L$ и элемент $x_0 \notin L$ на расстоянии $d > 0$ от $L$ (здесь $\displaystyle d = \inf_{x \in L}\Vert x_0 - x \Vert$). Тогда существует всюду определенный в $X$ линейный функционал $f$, для которого выполняется\cite[с.~167]{trenogin}:
\begin{enumerate}
	\itemsep0em
	\item $\langle x, f \rangle = 0$ для всех $x \in L$;
	\item $\langle x_0, f \rangle = 1$;
	\item $\Vert f \Vert = 1 / d$.
\end{enumerate}

\subsubsection*{Следствие}
Линейное многообразие $L$ не является плотным в банаховом пространстве $X$ тогда и только тогда, когда найдется линейный ограниченный функционал $f, f \neq 0$, такой, что $\langle x, f \rangle = 0$ для всех $x \in L$\cite[с.~168]{trenogin}.\\
Определение плотного множества: \hyperref[sec:q-17]{см. вопрос 17}.

\subsubsection*{Следствие}
Пусть $\{x_k\}^n_1$ --- линейно независимая система элементов в нормированном пространстве $X$. Тогда найдется система линейных, определенных на всём $X$, ограниченных функционалов $\{f_l\}^n_1$, такая, что $\langle x_k, f_l \rangle = \delta_{kl}\; (k,l = 1, \dots, n)$.

Здесь $\delta_{kl}$ определяется по формуле
$$\delta_{kl} =
\begin{cases}
	1,&k=l,\\
	0,&k\neq l.
\end{cases}$$

Система элементов $\{x_k\}^n_1$ и система функционалов $\{f_l\}^n_1$ называются \textit{биортогональными}, если
$$\langle x_k, f_l \rangle = \delta_{kl}, \quad k,\:l = 1, \dots, n.$$

\subsubsection*{Лемма}
Пусть дана линейно независимая система линейных функционалов $\{f_l\}^n_1$, которые определены и ограничены всюду на нормированном пространстве $X$. Тогда в $X$ найдется система элементов $\{x_k\}^n_1$, биортогональная с системой $\{f_l\}^n_1$\cite[с.~168]{trenogin}.