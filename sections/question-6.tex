\section{Теорема об эквивалентности норм}

Пусть в линейном пространстве $E$ введены две нормы: $\Vert \cdot \Vert_1$ и $\Vert \cdot \Vert_2$. Если существует число $\beta > 1$, такое, что для любых $x \in E$ выполняется
$$\Vert \cdot \Vert_1 \leqslant \beta \Vert \cdot \Vert_2,$$
то будем говорить, что норма $\Vert \cdot \Vert_1$ \textit{подчинена} норме $\Vert \cdot \Vert_2$.

Пусть $E$ --- линейное пространство и в $E$ двумя способами введены нормы: $\Vert x \Vert^{(1)}$ и $\Vert x \Vert^{(2)}$. Эти нормы называются \textit{эквивалентными}, если существуют числа $\alpha > 0, \beta > 0$ такие, что для любых $x \in E$ выполняется
$$\alpha\Vert x \Vert^{(1)} \leqslant \Vert x \Vert^{(2)} \leqslant \Vert x \Vert^{(1)}.$$
Эквивалентность норм можно определить иначе: две нормы в линейном пространстве эквивалентны тогда и только тогда, когда каждая из них подчинена другой.

\subsubsection*{Теорема (об эквивалентности норм)}
Во всяком конечномерном пространстве все нормы эквиваленты. Доказательство доступно в учебнике \cite[с.~28]{trenogin}.