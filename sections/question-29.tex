\section{Самосопряженные операторы и их свойства}
\label{sec:q-29}
Пусть $H$ --- гильбертово пространство. Оператор $A \in L(H)$ называется \textit{самосопряженным (эрмитовым, симметрическим)}, если $A^* = A$, то есть если $A$ совпадает со своим сопряженным \cite[с.~181]{trenogin}.

Согласно этому определению $A$ --- самосопряженный, если для любых $x, y \in H$ выполняется $(Ax, y) = (x, Ay).$

Свойства самосопряженных операторов:
\begin{enumerate}
	\itemsep0pt
	\item Спектр (множество собственных чисел) самосопряженного оператора является вещественным.
	\item В унитарных конечномерных пространствах матрица самосопряжённого оператора является эрмитовой (эрмитова матрица --- квадратная матрица, элементы которой являются комплексными числами, и которая, будучи транспонирована, равна комплексно сопряжённой; в евклидовом пространстве матрица самосопряжённого оператора является симметрической).
	\item У эрмитовой матрицы всегда существует ортонормированный базис из собственных векторов — собственные векторы, соответствующие различным собственным значениям, ортогональны.
\end{enumerate}
Доказательства: см. \cite{hermit-operator}.

\subsubsection*{Теорема}
Пусть $A$ и $B$ --- самосопряженные операторы в $H$, а $\alpha$ и $\beta$ --- вещественные числа. Тогда $\alpha A + \beta B$ --- самосопряженный оператор в $H$.

\subsubsection*{Теорема}
Пусть операторы $A$ и $B$ --- самосопряженные. Оператор $AB$ является самосопряженным тогда и только тогда, когда $A$ и $B$ перестановочны ($AB = BA$).

\subsubsection*{Теорема}
Если $A$ самосопряжен, то число $(Ax, x)$ вещественно для любых $x \in H$.

\subsubsection*{Теорема}
Если $A$ --- самосопряженный, то
$$\Vert A \Vert = \sup_{\Vert x \Vert \leqslant 1} | (Ax, x)|.$$
Доказательства теорем приводятся в учебнике \cite[с.~181]{trenogin}.