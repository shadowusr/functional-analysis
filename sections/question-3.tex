\section{Теорема Бэра о категориях}
Множество $M$ в метрическом пространстве $X$ называется \textit{нигде не плотным} в $X$, если в каждом шаре $S \subset X$ содержится другой шар $S_1$, не содержащий точек $M$ (метрическое пространство, шар: \hyperref[sec:q-1]{см. вопрос 1}).

Множество в метрическом пространстве называется \textit{множеством I категории}, если его можно представить в виде объединения счетного числа нигде не плотных множеств.

Если множество не является множеством I-й категории, оно является \textit{множеством II-й категории}.

\subsubsection*{Теорема (Бэра)} 
Всякое полное метрическое пространство является множеством II-й категории.\\
Доказательство: см. \cite[с.~49]{trenogin}.

\subsubsection*{Теорема (принцип вложенных шаров)}
Пусть в банаховом пространстве $X$ дана последовательность шаров $\{\bar{S_{r_n}}(x_n)\}$, вложенных друг в друга, то есть $\{\bar{S_{r_{n + 1}}}(x_{n + 1})\} \subset \{\bar{S_{r_n}}(x_n)\} \quad (n = 1, 2, \dots)$, причем $r_n \to 0$ при $n \to \infty$. Тогда в $X$ существует единственная точка $x$, принадлежащая всем шарам.\\
Доказательство:  см. \cite[с.~49]{trenogin}.