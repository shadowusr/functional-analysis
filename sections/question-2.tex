\section{Полные метрические пространства}
\label{sec:q-2}
Нормированное пространство (\hyperref[sec:q-5]{см. вопрос 5}) называется \textit{полным}, если в нем любая фундаментальная последовательность сходится (имеет предел, принадлежащий этому пространству). Полное нормированное пространство ещё называют \textit{банаховым} \cite[с.~42]{trenogin}.

\textit{Евклидово пространство} --- это конечномерное линейное пространство над полем вещественных чисел, на парах векторов которого задана вещественнозначная функция $(\cdot, \cdot)$, обладающая следующими тремя свойствами:
\begin{enumerate}
	\itemsep0pt
	\item Билинейность: для любых векторов $u,v,w$ и для любых вещественных чисел $a, b$ выполняется $(au+bv, w)=a(u,w)+b(v,w)$ и $(u, av+bw)=a(u,v)+b(u,w);$
	\item Симметричность: для любых векторов $u,v$ справедливо $(u,v)=(v,u);$
	\item Положительная определённость: для любого $u$ выполняется $ (u,u)\geqslant 0,$ причём $(u,u) = 0\Rightarrow u=0.$
\end{enumerate}

Говорят, что линейное пространство $X$ есть \textit{прямая сумма} своих подпространств $M_1, \dots, M_n$:$X = M_1 \oplus \dots \oplus M_n,$ если каждый вектор $x \in X$ единственным образом представляется в виде суммы
$x = m_1 + \dots + m_n, \qquad m_i \in M_i.$


\subsubsection*{Примеры банаховых пространств}
\begin{itemize}
	\itemsep0pt
	\item Евклидовы пространства $E^n$.
	\item Пространство непрерывных на $[a, b]$ функций $C[a, b]$. Этот факт можно показать при помощи следующей теоремы: если последовательность непрерывных на $[a, b]$ функций $\{x_n(t)\}$ сходится равномерно на $[a, b]$ к некоторой функции $x(t)$, то $x(t)$ непрерывна на $[a, b]$ \cite[с.~43]{trenogin}.
	\item Пространство всех бесконечных последовательностей $(x_1, x_2, \dots)$ таких, что ряд $ \sum_{i = 1}^{\infty}|x_i|^p$, где $p \geqslant 1$, сходится, является банаховым. Норма вводится как $ \sqrt{\sum_{i = 1}^{\infty}|x_i|^p}$.
	\item Если $X$ и $Y$ --- банаховы пространства, то их прямая сумма $X \oplus Y$ тоже является банаховым пространством.
	\item Любое гильбертово пространство тоже является банаховым. Обратное неверно.
\end{itemize}
Больше примеров: см. \cite{wiki-banach-space}.