\section{Операторы Гильберта--Шмидта}
\label{sec:q-40}
\textit{Оператор Гильберта--Шмидта} --- это ограниченный оператор $A$ в гильбертовом пространстве $H$ с конечной нормой Гильберта--Шмидта:
$$\Vert A \Vert_{HS}^2 = \sum_{i \in I} \Vert A e_i \Vert^2 < \infty,$$
где $\Vert \cdot \Vert$ --- норма пространства $H$, $\{e_i : i \in I\}$ --- ортонормированный базис в $H$. Здесь $I$ --- некоторое множество индексов.

Если это верно в каком-то ортонормированном базисе, то это верно в любом ортонормированном базисе.

Определение оператора Гильберта--Шмидта из учебника \cite[с.~460]{kolmogorov}:\\
Оператор $A$, определяемый равенством $A \varphi = \psi$ вида
$$\int\displaylimits_{a}^{b}K(s, t) \varphi(t)dt = \psi(s)$$
называется \textit{оператором Гильберта--Шмидта}, если его ядро $K(s, t)$ удовлетворяет условию
$$\int\displaylimits_a^b\int\displaylimits_a^b|K(s, t)^2 ds\;dt < \infty.$$

\textit{След матрицы} --- это сумма элементов на главной диагонали матрицы. Операция следа матрицы определена для квадратных матриц и обозначается $\operatorname{tr} A$.

Пусть $A$ и $B$ — два оператора Гильберта--Шмидта. Тогда \textit{скалярное произведение Гильберта--Шмидта} определяется как
$$\langle A,B \rangle_\mathrm{HS} = \operatorname{tr}\,A^TB
= \sum_{i \in I} \langle Ae_i, Be_i \rangle,$$
где $\operatorname{tr}$ обозначает след оператора.

Свойства:
\begin{itemize}
	\itemsep0pt
	\item Любой оператор Гильберта--Шмидта $T : H \to H$ является компактным.
	\item Если $T : H \to H$ --- ограниченный линейный оператор, то $\Vert T \Vert \leqslant \Vert T \Vert_{HS}$.
	\item Пусть $T : H \to H$ --- ограниченный линейный оператор и $S : H \to H$ --- оператор Гильберта--Шмидта, тогда
	$$\Vert S^* \Vert_{HS} = \Vert S \Vert_{HS}, \qquad \Vert TS \Vert_{HS} \leqslant \Vert T \Vert \Vert S \Vert_{HS}, \qquad \|ST\|_{HS} \leqslant \|S\|_{HS}\|T\|.$$
	\item Композиция двух операторов Гильберта--Шмидта также является оператором Гильберта--Шмидта.
\end{itemize}

% todo: Выписать конспект теоремы
Информация о теореме Гильберта--Шмидта: см. \cite{hilbert-schmidt-theorem}.