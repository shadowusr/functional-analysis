\section{Унитарные пространства. Гильбертовые пространства. Примеры}
\label{sec:q-22}
Комплексное линейное пространство $U$ называется \textit{унитарным}, если каждой паре его элементов $x$ и $y$ поставлено в соответствие комплексное число $(x, y)$ --- скалярное произведение $x$ на $y$ --- и если при этом выполняются следующие аксиомы:
\begin{enumerate}
	\itemsep0em
	\item $(x, x) \geqslant 0,\; (x, x) = 0$ при $x = 0$ и только в этом случае;
	\item $(x, y) = \overline{(y, x)}$ (черта означает комплексное сопряжение);
	\item $(\lambda x, y) = \lambda(x, y)$;
	\item $(x + y, z) = (x , z) + (y, z)$.
\end{enumerate}
Более подробная информация доступна в учебнике \cite[с.~37]{trenogin}.\\

\textit{Гильбертово пространство} --- это линейное пространство, такое, что:
\begin{enumerate}
	\itemsep0pt
	\item указано правило, которое позволяет определить для любых двух элементов пространства $x$ и $y$ их скалярное произведение $(x,y)$, причем должно выполняться:
	\begin{itemize}
		\item $(y, x) = (x, y)$ (переместительный закон в вещественном гильбертовом пространстве) или $(y, x) = \overline{(x, y)}$ (переместительный закон в комплексном гильбертовом пространстве, черта означает знак комплексного сопряжения);
		\item $(x, y + z) = (x, y) + (x, z)$ (распределительный закон);
		\item $(\lambda x, y) = \lambda (x, y)$ для любого вещественного числа $\lambda$;
		\item $(x, x) > 0$ при $x \ne 0$ и $(x, x)=0$ при $x=0$.
	\end{itemize}
	\item оно является полным относительно порождённой этим скалярным произведением метрики $d(x,y)=\|x-y\|=\sqrt{(x-y,x-y)}$. Если условие полноты пространства не выполнено, то говорят о \textit{предгильбертовом пространстве}. Однако большинство из известных (используемых) пространств либо являются полными, либо могут быть пополнены.
\end{enumerate}
Гильбертово пространство --- это обобщение Евклидова пространства, допускающее бесконечную размерность.\\
В учебнике приводится дополнительная информация о Гильбертовых пространствах\cite[с.~50]{trenogin}.

\subsubsection*{Примеры унитарных пространств}
\begin{itemize}
	\itemsep0pt
	\item Евклидово пространство $E^m$. Скалярное произведение в вещественном линейном пространстве $E^m$ введем по формуле
	$$(x, y) = \sum_{k = 1}^{m}\xi_k\eta_k.$$
	\item Пространство $l_2$. Его точки суть бесконечные последовательности вещественных чисел $x = \{x_n\}_{n=1}^{\infty}$, для которых сходится ряд $\sum_{n=1}^\infty |x_n|^2$. Скалярное произведение на этом пространстве задаётся равенством
	$$(x, y) = \sum_{n=1}^\infty x_n y_n.$$
	\item Пространство $\tilde{L}_2[a, b]$ --- линейное пространство комплекснозначных, непрерывных на $[a, b]$ функций. Скалярное произведение зададим так:
	$$(x, y) = \int_{a}^{b}x(t)\overline{y(t)} dt.$$
\end{itemize}

\subsubsection*{Примеры Гильбертовых пространств}
\begin{itemize}
	\itemsep0pt
	\item Пространства $E^m$ и $l_2$ из предыдущего раздела.
	\item Пространство $L^2[a,b]$ измеримых функций с вещественными значениями на отрезке $[a,b]$ с интегрируемыми по Лебегу квадратами --- то есть таких, что интеграл
	$$\int\limits_a^b|f|^2\,dx$$
	определён и конечен. Скалярное произведение на этом пространстве задаётся равенством
	$$(f, g) = \int\limits_a^bf{g}\,dx$$.
\end{itemize}
Подробнее об измеримых функциях: \cite{measurable-function}.\\
Больше примеров гильбертовых пространств: \cite{hilbert-spaces}.