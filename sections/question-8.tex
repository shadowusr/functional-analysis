\section{Классификация отображений. Линейные отображения}
\label{sec:q-8}
\textit{Отображение} --- это соответствие между элементами двух множеств, установленное таким образом, что каждому элементу первого множества соответствует ровно один элемент второго множества.\\
Формально, \textit{отображением} называется множество упорядоченных пар $(x,y)\in X\times Y$, таких, что пары существуют для всех элементов множества $X$, и, если первые элементы пар совпадают, то совпадают и вторые их элементы.

Существуют различные варианты классификации отображений, здесь приводится вариант, изложенный на лекции.

Отображение называется \textit{функцией}, если $X$ и $Y$ --- числовые множества.

Если $Y$ ---  числовое, но $X$ не является числовым, то такое отображение называется \textit{функционалом}.

Если $X$ и $Y$ --- не числовые множества, то отображение называется \textit{оператором}.

\textit{Линейным отображением} линейного пространства $V$ в линейное пространство $W$ над полем $K$ называется отображение $f\colon V\rightarrow W$, удовлетворяющее условию линейности:
\begin{itemize}
	\itemsep0em
	\item $f(x+y)=f(x)+f(y)$,
	\item $f(\alpha x)=\alpha f(x)$
\end{itemize}
для всех ${\displaystyle x,y\in V}$ и $\alpha \in K$. Определения поля и линейного пространства: \hyperref[sec:q-5]{см. вопрос 5}.

Если $V$ и $W$ --- это одно и то же векторное пространство, то $f$ --- не просто линейное отображение, а \textit{линейное преобразование}.

Более подробная информация о линейных операторах доступна в учебнике \cite[с.~109]{trenogin}.