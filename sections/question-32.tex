\section{Классификация интегральных уравнений}
\label{sec:q-32}
\textit{Интегральным уравнением} называется уравнение, содержащее неизвестную функцию, которую обозначим $y(x)$, под знаком интеграла:

$$\int\displaylimits_{a}^{b}F(x, t, y(t)) dt = G(x, y(x)).$$
Здесь $F, G$ --- заданные функции. Если функция $G$ в правой части уравнения не зависит от $y(x)$, то говорят об интегральном уравнении \textit{1-го рода}, в общем случае --- об уравнении \textit{2-го рода}.

Область $S = [a, b] \times [a, b]$ изменения переменных $x$ и $t$ в уравнении выше называется \textit{основным квадратом}. Промежуток $[a, b]$ называется \textit{областью определения} уравнения.

$Решение интегрального уравнения$ на промежутке $[a, b]$ --- это функция $y(x)$, которая при подстановке в это уравнение обращает его в тождество для всех $x \in [a, b]$.

Интегральное уравнение называют \textit{линейным}, если неизвестная функция входит в него линейно.

Линейное интегральное уравнение вида
$$\int\displaylimits_{a}^{b}K(x, t) y(t) dt = f(x).$$
называют \textit{интегральным уравнением Фредгольма 1-го рода}.

Уравнение
\begin{equation}
	y(x) = \lambda \int\displaylimits_{a}^{b}K(x, t)y(t)dt + f(x) \label{eq:1}
\end{equation}
называется \textit{интегральным уравнением Фредгольма 2-го рода}.

Здесь функции $f(x)$, $K(x,t)$ являются заданными, $y(x)$ --- искомая функция, $\lambda$ --- числовой параметр, $x \in [a,b], t \in [a, b]$.

Функция $K(x, t)$ называется \textit{ядром интегрального уравнения}, $f(x)$ --- \textit{свободным членом}. Если $f = 0$, то уравнение называется \textit{однородным}, иначе --- \textit{неоднородным}.

В уравнениях Фредгольма ядро $K(x, t)$ и свободный член $f(x)$ либо непрерывны ($K(x, t)$ в квадрате $S$, а $f(x)$ --- на отрезке $[a, b]$), либо квадратично интегрируемы, т. е. удовлетворяют условиям:
$$\int\displaylimits_a^b\int\displaylimits_a^b|K(x, t)|^2dxdt < + \infty; \qquad \int\displaylimits_a^b|f(x)|^2 dx < + \infty.$$

Ядра, удовлетворяющие условию выше, называют \textit{фредгольмовыми}.

Ядра, для которых выполняется условие
$$K(x, t) = 0, t > x.$$
называют \textit{ядрами Вольтерры}, а уравнения
$$\int\displaylimits_a^x K(x, t)y(t)dt = f(x),$$
$$y(x) = \lambda \int\displaylimits_a^xK(x, t) y(t) dt + f(x).$$
называют \textit{уравнениями Вольтерры} соответственно \textit{1-го} и \textit{2-го рода}.

Уравнение
$$z(x) = \lambda^* \int\displaylimits_a^b K^*(t, x)z(t)dt + g(x),$$
где $\lambda^*, K^*$ --- комплексно сопряженные к $\lambda, K$ величины, называется уравнение, \textit{союзным (сопряженным)} к уравнению (\ref{eq:1}), а ядро $K^*(t, x)$ --- союзным к ядру $K(x, t)$.

Если $K(x, t) = K^*(t, x)$, то ядро интегрального уравнения называется \textit{эрмитовым}.

Если ядро вещественно и $K(x, t) = K(t, x)$, то ядро называется \textit{симметричным}.

Ядро называется \textit{полярным}, если оно представимо в виде
$$K(x, t) = \frac{\Phi(x, t)}{|x - t|^\alpha},$$
где $\Phi(x, t)$ --- непрерывная функция, $\alpha$ --- число, $0 < \alpha < n, n$ --- размерность пространства переменных (при $x \in [a, b]$ размерность $n = 1$).

Если $\alpha < n / 2$, то такое ядро называется \textit{слабополярным}.

Ядро называется \textit{вырожденным}, если оно представимо в виде конечной суммы произведений функций, каждая из которых зависит только от одной переменной:
$$K(x, t) = \sum_{i = 1}^{n} \Omega_i(x)\omega_i(t)$$
Можно считать функции $\Omega_i(x)$ и $\omega_i(x)$ линейно независимыми, т. к. каждая линейная связь между ними уменьшает число слагаемых в сумме на 1.

Примеры, доказательства и более подробная информация: см. \cite[с.~5]{int-diff-equations}.