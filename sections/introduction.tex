\section*{Вместо интро}
\addcontentsline{toc}{section}{\protect\numberline{}Интро}
Привет!

На этих страницах --- все лекционные материалы по функциональному анализу за 3 курс, написанные максимально понятным и простым языком, но с сохранением подробностей и строгости изложения. Доказательства теорем не приводятся, вместо этого каждая теорема сопровождается ссылкой на источник, где можно ознакомиться с доказательством.

Всем понятиям, которые используются в конспекте, дано определение, исключая необходимость обращаться к внешним источникам.

Значительная часть материалов взята из учебника <<Функциональный анализ>> Треногина~В.~А.\cite{trenogin}, хотя в целом конспект собирался из большого количества учебников, статей, а иногда --- из отдаленных уголков Всемирной сети.

Учебник <<Функциональный анализ>> доступен по ссылке: \url{https://vk.cc/av3dPG}.

Конспект создан с помощью \LaTeX, исходные файлы доступны на всем известном Hub'е.

Мы высоко ценим помощь в создании конспекта. При желании исправить ошибку, дополнить или уточнить материал, написать новый раздел или улучшить оформление предлагаем обратиться к репозиторию на GitHub: \url{https://github.com/shadowusr/functional-analysis}.

Сообщить об ошибке можно в разделе <<issues>>, либо открыв pull--request и исправив её самостоятельно.

В этом же репозитории в разделе <<релизы>> будут появляться новые версии этого pdf--файла.

Кто дочитал --- вэлкам на лобби с зяблами. 5х5. Остальные --- пожилые кроллы.

\begin{flushright}
	\textit{--- Чё за умник это писал вообще, лол}
\end{flushright}