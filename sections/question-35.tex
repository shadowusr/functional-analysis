\section{Предкомпактные множества. Теорема Хаусдорфа}
\label{sec:q-35}

\textit{Замыкание множества} --- это наименьшее замкнутое множество, содержащее в себе данное.

Множество $M$ называется \textit{предкомпактным} (или \textit{компактным относительно топологического пространства} $T$), если его замыкание в $T$ компактно.

Пусть $X$ --- нормированное пространство, $A, B \subset X, \; \varepsilon > 0$. Будем называть $B$ \textit{$\varepsilon$--сетью} для $A$, если для любого $a \in A$ найдётся $b \in B$ такой, что $\Vert a, b\Vert < \varepsilon$.

Если $B$ состоит из конечного числа элементов, будем говорить, что $\varepsilon$--сеть \textit{конечна}.

\subsubsection*{Теорема (Хаусдорфа)}
Множество $M$ в нормированном пространстве $X$ компактно тогда и только тогда, когда для любого $\varepsilon > 0$ в $X$ существует конечная $\varepsilon$--сеть.\\
Доказательство: см. учебник \cite[с.~195]{trenogin}.

\subsubsection*{Замечание}
Пусть $X$ --- банахово пространство, а $M \subset X$ --- замкнуто. Если для любого $\varepsilon > 0$ существует конечная $\varepsilon$--сеть множества $M$, то $M$ бикомпактно.