\section{Непрерывные и ограниченные отображения}
\label{sec:q-9}
Пусть дано отображение $F : X \rightarrow Y$, определенное в окрестности точки $x_0$. Оператор $F$ называется \textit{непрерывным в точке $x_0$}, если $F(x)\rightarrow F(x_0)$ при $x \rightarrow x_0$.

\subsubsection*{Теорема}
Пусть линейный оператор $A$, заданный в банаховом пространстве $X$ и со значениями в банаховом пространстве $Y$, непрерывен в точке $0 \in X$, тогда $A$ непрерывен в любой точке $x_0 \in X$\cite[c.~112]{trenogin}.\\

Будем называть линейное отображение $A:X\to Y$ в нормированном пространстве \textit{ограниченным}, если существует такое положительное число $C$, что $\|Ax\| \leqslant C\|x\|$ для любого $x$ из $D(A)$.

Более подробная информация доступна в учебнике \cite[c.~112]{trenogin}.