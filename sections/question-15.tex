\section{Обратимость линейных операторов}
\textit{Обратный оператор} $A^{-1}$ к оператору $A$ --- оператор, который каждому $y$ из множества значений $R(A)$ оператора $A$ ставит в соответствие единственный элемент из области определения $D(A)$ оператора $A$, являющийся решением уравнения $Ax = y$.

Другое определение: оператор $A^{-1}$ называется \textit{обратным} к оператору $A$, если ${\displaystyle A^{-1}A=I,\,AA^{-1}=I}$, где $I$ --- единичный оператор. Если выполняется только соотношение ${\displaystyle A^{-1}A=I}$ или только ${\displaystyle AA^{-1}=I,}$ то оператор $A^{-1}$ называется \textit{левым обратным} или \textit{правым обратным} соответственно.

Если оператор $A$ имеет обратный, то есть уравнение ${\displaystyle Ax=y}$ имеет единственное решение при любом $y$ из $R(A)$, то $A$ называется \textit{обратимым}.

Оператор $A$ обратим, если он отображает ${\displaystyle D(A)}$ на ${\displaystyle R(A)}$ взаимно однозначно, то есть при различных ${\displaystyle x\in D(A)}$ принимает различные значения $y$.

Если оператор $A$ --- линейный, то для существования обратного оператора достаточно, чтобы $A x = 0$ выполнялось только при $x = 0$.

Будем говорить, что линейный оператор $A:X\rightarrow Y$ \textit{непрерывно обратим}, если существует обратный оператор $A^{-1} : Y \rightarrow X$, который ограничен и определен на всём $Y$.

\subsubsection*{Теорема}
Зададим множество нулей оператора $A$ следующим образом:
$$N(A)=\{x \in DA(A) : Ax=0\}.$$
Оператор $A$ переводит $D(A)$ в $R(A)$ взаимно однозначно тогда и только тогда, когда множество нулей оператора $A$ состоит только из элемента $0$ (т.е. $N(A) = \{0\}$) \cite[с.~127]{trenogin}.

\subsubsection*{Теорема}
Оператор $A^{-1}$ существует и одновременно ограничен на $R(A)$ тогда и только тогда, когда для некоторой постоянной $m > 0$ и любого $x \in D(A)$ выполняется $\Vert Ax \Vert \geqslant m \Vert x \Vert$\cite[с.~127]{trenogin}.\\
Напомним, что линейный оператор $A:X\to Y$ в нормированном пространстве называется \textit{ограниченным}, если существует такое положительное число $C$, что $\|Ax\| \leqslant C\|x\|$ для всех $x \in D(A)$.

\subsubsection*{Теорема}
Оператор $A$ непрерывно обратим тогда и только тогда, когда $R(A) = Y$ и для некоторой постоянной $m > 0$ и всех $x \in D(A)$ выполняется $\Vert Ax \Vert \geqslant m \Vert x \Vert$\cite[с.~128]{trenogin}.

\subsubsection*{Теорема}
Если $A$ --- ограниченный линейный оператор, взаимно однозначно отображающий $X$ на $Y$, причем $X$ и $Y$ --- банаховы пространства, то обратный оператор $A^{-1}$ ограничен\cite[с.~128]{trenogin}. Подробнее о банаховых пространствах: \hyperref[sec:q-2]{см. вопрос 2}.