\section{Ряды Фурье в гильбертовом пространстве}
\label{sec:q-24}
\textit{Ортогональная система} элементов векторного пространства со скалярным произведением --- это такое подмножество векторов $\left\{ \varphi_i \right\}\subset H$ ($H$ --- гильбертово пространство), что любые различные два из них ортогональны, то есть их скалярное произведение равно нулю: $(\varphi_i, \varphi_j) = 0$.

Ряды Фурье позволяют выполнить приближение некоторой периодической функции путем представления её в виде набора синусоид, умноженных на некоторый коэффициент.\\
Ознакомиться с формальным определением рядов Фурье и формулами для общего случая можно на википедии \cite{fourier-series}.

Пусть в гильбертовом пространстве $H$ дана ортогональная система $\allowbreak \{\varphi_1, \varphi_2,\allowbreak\dots, \varphi_n, \dots\}$ и $x$ --- произвольный элемент из $H$. Предположим, что мы хотим представить $x$ в виде (бесконечной) линейной комбинации элементов $\{\varphi_k\}$: 
$$x = \sum^{\infty}_{n=1}c_n\varphi_n.$$
Домножим это выражение на $\varphi_k$. С учётом ортогональности системы функций $\{\varphi_k\}$ все слагаемые ряда обращаются в ноль, кроме слагаемого при $n=k$:
$$(x, \varphi_k) = c_k\|\varphi_k\|^2.$$
Числа 
$$c_k =\frac{(x, \varphi_k)}{\|\varphi_k\|^2}$$
называются \textit{координатами}, или \textit{коэффициентами Фурье} элемента $x$ по системе $\{\varphi_k\}$, а ряд
$$\sum_k c_k \varphi_k$$
называется \textit{рядом Фурье} элемента $x$ по ортогональной системе $\{\varphi_k\}$.

Многочлен $\displaystyle \sum_{k=1}^{n}c_k\varphi_k$ --- частичная сумма ряда Фурье --- называется \textit{многочленом Фурье} элемента $x$.

Ряд Фурье любого элемента $x$ по любой ортогональной системе сходится в пространстве $H$, но его сумма не обязательно равна $x$.

Дополнительная информация доступна в учебнике \cite[с.~54]{trenogin}.