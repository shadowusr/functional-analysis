\section{Решение интегрального уравнения Фредгольма 2--го рода методом итерированных ядер}
\label{sec:q-33}
Линейное неоднородное уравнение Фредгольма 2--го рода имеет вид
$$y(x) = \lambda \int\displaylimits_{a}^{b}K(x, t) y(t) dt + f(x), \; x \in [a, b].$$
Здесь $K(x, y)$ --- ядро уравнения, $f(x)$ --- непрерывная функция.

Решение уравнений такого вида может быть получено методом итерированных ядер (методом резольвент). Для этого используется формула
$$y(x) = f(x) + \lambda \int\displaylimits_{a}^{b} R(x, t, \lambda)f(t)dt,$$
где функция $R(x, t, \lambda)$ называется \textit{резольвентой} ядра $K(x, t)$ и определяется как сумма ряда
$$R(x, t, \lambda) = \sum_{n = 1}^{\infty} \lambda^{n - 1}K_n(x, t),$$
где $K_n(x, y)$ --- повторные (итерированные) ядра, определяемые из рекуррентного соотношения
$$K_1(x, t) = K(x, t), \qquad K_{n + 1} = \int\displaylimits_{a}^{b} K(x, s)K_n(s, t)ds, \; n = 1, 2, \dots $$
% todo: Написать пример
Пример решения уравнения этим методом: \cite[с.~34]{int-diff-equations}.\\
Подробнее об итерированных ядрах: \cite{iter-kernels}.