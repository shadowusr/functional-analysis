\section{Компактные множества и их свойства}
\label{sec:q-34}
% todo: систематизировать раздел. 
В математическом анализе существенную роль играет теорема Больцано--Вейерштрасса, в которой утверждается: из любой ограниченной последовательности вещественных чисел можно выделить сходящуюся подпоследовательность. Информация ниже затрагивает свойство, выражаемое этой теоремой, в случае бесконечномерных пространств.

Напомним, что \textit{подпоследовательностью} последовательности $\{x_n\}$ называется её подмножество $\{x_{n_k}\}$, если $n_{k+1} > n_k, \ (k = 1, 2, \dots)$, т.е. если в $\{x_{n_k}\}$ сохраняется порядок следования элементов $\{x_n\}$.

Множество $M$ банахова пространства $X$ называется \textit{бикомпактным}, если из каждой последовательности $\{x_n\} \subset M$ можно выбрать сходящуюся подпоследовательность, предел которой принадлежит $M$.

Из этого определения вытекает ряд следствий:
\begin{itemize}
	\itemsep0pt
	\item Всякое бикомпактное множество ограничено.
	\item Всякое бикомпактное множство замкнуто.
\end{itemize}
Множество $M$ нормированного пространства $X$ называется \textit{компактным}, если из каждой последовательности $\{x_n\} \subset M$ можно выделить фундаментальную подпоследовательность.

Замкнутое подмножество компактного множества является компактным.

Конечное объединение компактных множеств является компактным.

Декартово произведение произвольного набора компактных множеств является компактным.

\subsubsection*{Теорема}
Пусть $f(x)$ --- вещественный нелинейный непрерывный функционал, определенный на бикомпактном множестве $Q$; тогда $f(x)$ ограничен на $Q$.

\subsubsection*{Теорема}
Пусть $f(x)$ --- вещественный нелинейный непрерывный функционал, определенный на бикомпактном множестве $Q$. Тогда $f(x)$ достигает на $Q$ своих наименьшего и наибольшего значений.

\subsubsection*{Теорема}
Компактное множество в нормированном пространстве бикомпактно тогда и только тогда, когда оно замкнуто.

Доказательства теорем и подробная информация: см. учебник \cite[с.~194]{trenogin}.