\section{Расширение линейных отображений с сохранением нормы}

Отображение $\Phi$ называется \textit{расширением} отображения $F$ (а отображение $F$ --- \textit{сужением} отображения $\Phi$), если $D(\Phi) \supset D(F)$ и $\Phi(x) = F(x)$ для всех $x \in D(F)$ \cite[с.~109]{trenogin}.

Отображение $\Phi$ называется \textit{расширением $F$ с сохранением нормы}, если $\Vert \Phi \Vert = \Vert F \Vert$ (и $\Phi$ --- расширение $F$).

Множество $\widetilde{E}$ в линейном пространстве $E$ называется \textit{линейным многообразием}, если для любых $x, y \in \widetilde{E}$ и любых скаляров $\lambda, \mu$ линейная комбинация $\lambda x + \mu y$ принадлежит $\widetilde{E}$ \cite[с.~13]{trenogin}.

\subsubsection*{Теорема (Хана---Банаха)}
Всякое линейное ограниченное отображение $F(x)$, определённое на линейном многообразии $L$ линейного нормированного пространства $X$, можно продолжить на всё пространство с сохранением нормы \cite[с.~163]{trenogin}.

Альтернативная формулировка: пусть в вещественном нормированном пространстве $X$ задано линейное ограниченное отображение $F$ с $D(F) \subset X$ (то есть область определения $F$ --- некоторое подмножество $X$). Тогда существует линейное ограниченное отображение $\Phi$, которое определено на всём $X$ и выполняется $\Vert \Phi \Vert = \Vert F \Vert$, $\Phi(x) = F(x)$ для всех $x \in D(F)$.