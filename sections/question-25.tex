\section{Изоморфизм сепарабельных гильбертовых пространств}
\label{sec:q-25}
Подмножество $A$ топологического пространства $X$ называется \textit{плотным}, если в произвольной окрестности каждой точки $x \in X$ содержится элемент из $A$.

Нормированное пространство $X$ называется \textit{сепарабельным}, если в нём существует счетное, плотное в $X$ множество.

Последовательность $\{e_k\}$ элементов нормированного пространства $X$ называется \textit{базисом} в $X$, если для любого $x \in X$ существует единственная последовательность чисел $\lambda_k, (k \in \mathbb{N})$, таких, что
$$x = \sum_{k = 1}^{\infty}\lambda_k e_k.$$

Любое линейное взаимно однозначное отображение линейного пространства $X$ на линейное пространство $Y$ называется \textit{изоморфизмом} этих пространств. В этом случае пространства $X, Y$ называются \textit{изоморфными}.

\subsubsection*{Теорема}
Во всяком сепарабельном евклидовом пространстве $E$ существует ортонормированный базис (т. е. скалярное произведение любых двух элементов из базиса равно нулю).\\
Доказательство: \cite[с.~91]{mipt-lectures-1}.

\subsubsection*{Теорема}
Любое сепарабельное бесконечномерное гильбертово пространство $H$ изоморфно пространству $l_2$ числовых последовательностей.\\
Определение пространства $l_2$:  \hyperref[sec:q-23]{см. вопрос 23}, примеры унитарных пространств.\\
Доказательство: \cite[с.~92]{mipt-lectures-1}.

\subsubsection*{Теорема}
Все сепарабельные бесконечномерные гильбертовы пространства изоморфны между собой.\\
Доказательство: \cite[с.~93]{mipt-lectures-1}.