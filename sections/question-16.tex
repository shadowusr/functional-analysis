\section{Множество регулярных точек оператора}
\label{sec:q-16}
Пусть $X$ --- комплексное банахово пространство. Будем рассматривать линейный оператор $A:X\rightarrow X$.

Точка $\lambda$ называется \textit{регулярной точкой} оператора $A$, если оператор $A - \lambda I$ непрерывно обратим, где $\lambda$ --- комплексное число, $I$ --- единичный оператор\cite[с.~247]{trenogin}.

Совокупность регулярных точек оператора $A$ называется \textit{резольвентным множеством} оператора и обозначается $\rho(A)$.

Линейный оператор $R_\lambda(A) = (A - \lambda I)^{-1}$ называется \textit{резольвентой} оператора $A$, если $\lambda \in \rho(A)$ и $R_\lambda(A) \in L(X)$. Здесь $L(X) = L(X, X)$ --- пространство линейных непрерывных операторов, подробнее:  \hyperref[sec:q-11]{см. вопрос 11}.

\subsubsection*{Теорема}
Резольвентное множество $\rho(A)$ всегда открыто \cite[с.~247]{trenogin}.\\
Напомним, что множество называется открытым, если вместе с каждой его точкой в него входит ещё и некоторая окрестность этой точки.

\subsubsection*{Теорема}
Пусть $A \in L(X)$. Тогда
$$\{\lambda : |\lambda| > \Vert A \Vert \} \subset \rho(A).$$
Следствие: если $A$ ограничен, то $\rho(A)$ неограничено\cite[с.~248]{trenogin}.