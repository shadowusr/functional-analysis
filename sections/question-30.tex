\section{Ядра Фредгольма и их произведение}
\label{sec:q-30}
Линейное интегральное уравнение вида
$$\int\displaylimits_{a}^{b}K(x, t) y(t) dt = f(x).$$
называют \textit{интегральным уравнением Фредгольма 1-го рода}.

Уравнение
\begin{equation}
y(x) = \lambda \int\displaylimits_{a}^{b}K(x, t)y(t)dt + f(x) \label{eq:1}
\end{equation}
называется \textit{интегральным уравнением Фредгольма 2-го рода}.

Здесь функции $f(x)$, $K(x,t)$ являются заданными, $y(x)$ --- искомая функция, $\lambda$ --- числовой параметр, $x \in [a,b], t \in [a, b]$.

Функция $K(x, t)$ называется \textit{ядром интегрального уравнения}, $f(x)$ --- \textit{свободным членом}. Если $f = 0$, то уравнение называется \textit{однородным}, иначе --- \textit{неоднородным}.

В уравнениях Фредгольма ядро $K(x, t)$ и свободный член $f(x)$ либо непрерывны ($K(x, t)$ в квадрате $S$, а $f(x)$ --- на отрезке $[a, b]$), либо квадратично интегрируемы, т. е. удовлетворяют условиям:
$$\int\displaylimits_a^b\int\displaylimits_a^b|K(x, t)|^2dxdt < + \infty; \qquad \int\displaylimits_a^b|f(x)|^2 dx < + \infty.$$

Ядра, удовлетворяющие условию выше, называют \textit{фредгольмовыми}.

Если $K(x, t) = K^*(t, x)$, то ядро интегрального уравнения называется \textit{эрмитовым}.

Если ядро вещественно и $K(x, t) = K(t, x)$, то ядро называется \textit{симметричным}.

Ядро называется \textit{полярным}, если оно представимо в виде
$$K(x, t) = \frac{\Phi(x, t)}{|x - t|^\alpha},$$
где $\Phi(x, t)$ --- непрерывная функция, $\alpha$ --- число, $0 < \alpha < n, n$ --- размерность пространства переменных (при $x \in [a, b]$ размерность $n = 1$).

Если $\alpha < n / 2$, то такое ядро называется \textit{слабополярным}.

Ядро называется \textit{вырожденным}, если оно представимо в виде конечной суммы произведений функций, каждая из которых зависит только от одной переменной:
$$K(x, t) = \sum_{i = 1}^{n} \Omega_i(x)\omega_i(t)$$
Можно считать функции $\Omega_i(x)$ и $\omega_i(x)$ линейно независимыми, т. к. каждая линейная связь между ними уменьшает число слагаемых в сумме на 1.