\section{Следствия из теоремы Хаусдорфа}
\label{sec:q-36}
Подмножество $A$ топологического пространства $X$ называется \textit{плотным}, если в произвольной окрестности каждой точки $x \in X$ содержится элемент из $A$.

Топологическое пространство $X$ называется \textit{сепарабельным}, если оно содержит в себе счётное, плотное в $X$ подмножество.

\subsubsection*{Следствие}
Если для любого $\varepsilon > 0$ для множества $M$ существует компактная $\varepsilon$--сеть в $X$, то $M$ компактно.

\subsubsection*{Следствие}
Компактное множество ограничено.

\subsubsection*{Следствие}
Всякое компактное множество сепарабельно.\\
Доказательства следствий: см. учебник \cite[с.~196]{trenogin}.