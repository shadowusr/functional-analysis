\section{Теорема Рисса об общем виде линейного функционала в гильбертовом пространства}
\label{sec:q-26}
\subsubsection*{Теорема (Рисса об общем виде линейных функционалов в гильбертовом пространстве)}
Пусть $H$ --- гильбертово пространство. Для любого линейного ограниченного функционала $f$, заданного всюду на $H$, существует единственный элемент $y \in H$ такой, что для всех $x \in H$
$$\langle x, f \rangle = (x, y).$$
При это $\Vert f \Vert = \Vert y \Vert$.\\
Доказательство есть в учебнике \cite[с.~171]{trenogin}.

Теорема Рисса указывает на возможность установления взаимно однозначного соответствия между пространствами $H$ и $H^*$, сохраняющего норму.

С точностью до этого взаимно однозначного соответствия можно принять $H^* = H$, то есть пространство, сопряженное к гильбертову пространству, <<совпадает>> с $H$. В этом смысле можно говорить о самосопряженности гильбертова пространства.