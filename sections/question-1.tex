\section{Метрические пространства}\label{sec:q-1}
\textbf{Метрическое пространство} ~--- это непустое множество, в котором между любой парой элементов определено расстояние, называемое \textit{метрикой}\cite[с.~18]{trenogin}.

Формально, метрическое пространство есть пара $(X, d)$, где $X$ --- множество, $d$ --- функция $d \colon M \times M \to \mathbb{R}$, такая, что для любых $x,y,z\in M$ выполняется:
\begin{enumerate}
	\itemsep0em
	\item ${\displaystyle d(x,y)=0\Leftrightarrow x=y}$ (аксиома тождества)
	\item ${\displaystyle d(x,y)=d(y,x)}$ (аксиома симметрии)
	\item ${\displaystyle d(x,z)\leqslant d(x,y)+d(y,z)}$ (аксиома треугольника)
	\item $d(x,y)\geqslant 0$ (неотрицательность; может быть получена из аксиом 1--3)
\end{enumerate}
% TODO: add examples of distance functions

Множество $S_r(x_0)$ называется \textit{открытым шаром} с центром в точке $x_0$, радиуса $r$ и определяется как $S_r(x_0)=\left\{x \in E\colon \left\Vert x - x_0 \right\Vert < r\right\}$, где $E$ --- некоторое метрическое пространство, $x_0 \in E$ --- фиксированная точка, а $r > 0$.

Аналогично, множество
$$\bar{S}_r(x_0)=\left\{x \in E\colon \left\Vert x - x_0 \right\Vert \leqslant r\right\}$$
называется \textit{замкнутым шаром} (с центром в $x_0$ радиуса $r$). Множество
$$\sigma_r(x_0)=\left\{x \in E\colon \left\Vert x - x_0 \right\Vert = r\right\}$$
называется \textit{сферой}. Очевидно, $\bar{S}_r(x_0)=S_r(x_0) \cup \sigma_r(x_0)$.\\

\textit{Топологическое пространство} --- это уморядоченная пара $(X, \tau)$, где $X$ --- множество, $\tau$ --- множество подмножеств $X$, удовлетворяющее аксиомам:
\begin{enumerate}
	\itemsep0em
	\item Пустое множество и само $X$ входят в $\tau$;
	\item Любое произвольное (конечное или бесконечное) объединение элементов $\tau$ тоже принадлежит $\tau$;
	\item Пересечение произвольного конечного числа элементов из $\tau$ тоже принадлежит $\tau$.
\end{enumerate}
Элементы множества $\tau$ называются \textit{открытыми множествами}, а $\tau$ называется \textit{топологией} на $X$. Исторически понятие топологического пространства появилось как обобщение метрического пространства.