\section{Ряды в банаховых пространствах}
Пусть $X$ --- нормированное пространство и $x_k \in X\;(k = 1, 2, \ldots)$. Формальная сумма $\displaystyle \sum_{k=1}^{\infty}x_k$ называется \textit{рядом} в $X$. Элементы $\displaystyle s_n = \sum_{k=1}^{n}x_k$ будем называть \textit{частичными суммами} ряда $\displaystyle \sum_{k=1}^{\infty}x_k$.

Ряд $\displaystyle \sum_{k=1}^{\infty}x_k$ называется \textit{сходящимся}, если в $X$ сходится последовательность его частичных сумм $\{s_n\}$. Если ряд сходится, то его частичная сумма $s_n \rightarrow s$ при $n \rightarrow \infty$, а элемент $s$ называется \textit{суммой ряда} $\displaystyle \sum_{k=1}^{\infty}x_k$. Запись $\displaystyle \sum_{k=1}^{\infty}x_k = s$ означает, что ряд сходится и его сумма равна $s$.

Если сходится числовой ряд $\displaystyle \sum_{k=1}^{\infty}\Vert x_k \Vert$, то говорят, что ряд $\displaystyle \sum_{k=1}^{\infty}x_k$ \textit{сходится абсолютно}.

\subsubsection*{Критерий Коши сходимости ряда}
Пусть $X$ --- нормированное пространство. Для того, чтобы ряд $\displaystyle \sum_{k=1}^{\infty}x_k$ сходился, необходимо, а если $X$ банахово, то и достаточно, чтобы для любого $\varepsilon > 0$ нашелся номер $N$ такой, что при всех $n > N$ и при всех натуральных $p$ выполнялось неравенство
$$\left\Vert \sum_{k=n+1}^{n+p}x_k \right\Vert < \varepsilon.$$
Доказательство доступно в учебнике \cite[с.~46]{trenogin}. Подробнее о нормированных пространствах: \hyperref[sec:q-2]{см. вопрос 2}; о банаховых пространствах: \hyperref[sec:q-5]{см. вопрос 5}.

\subsubsection*{Теорема}
Пусть $X$ --- банахово пространство. Тогда всякий абсолютно сходящийся в $X$ ряд сходится. Доказательство: \cite[с.~47]{trenogin}.

\subsubsection*{Теорема}
Если в нормированном пространстве каждый абсолютно сходящийся ряд сходится, то $X$ банахово. Доказательство: \cite[с.~47]{trenogin}.